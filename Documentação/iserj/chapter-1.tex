% -*- coding: utf-8; -*-

\chapter{Análise de Requisitos}

\section{Requisitos Funcionais}
\begin{itemize}
	\item Efetuar login;
	\item Realizar cadastro de controle de estoque;
	\item Minimizar ao máximo o número de entregas atrasadas; * Perguntar Vínicus
	\item Inserir dados cadastrais do cliente em formulário próprio;
	\item Permitir a busca itens como cafés, doces e salgados específicos contidos no cardápio da loja;
	\item Adicionar itens como cafés, doces e salgados específicos contidos no cardápio da loja;
	\item Receber pontos por compras realizadas;
	\item Trocar pontos por cupons na sessão "Descontos Especiais" do site;
	\item Permitir a utilização de cupons antes de efetuar os pedidos;
	\item Realizar pedidos, escolhendo a forma de pagamento desejada;
	\item Comunicar-se com a/o atendente da loja.
\end{itemize}

\section{Requisitos Não Funcionais}
\begin{itemize}
	\item O sistema deverá executar
		\begin{itemize}
			\item nos Sistemas Operacionais de 64 bits da sua versão estável mais antiga até sua versão superior
				\begin{itemize}
					\item Windows XP e mais atuais;
					\item Linux nas suas Distribuições Ubuntu e Debian;
					\item MacOS.
				\end{itemize}
			\item nos navegadores da versão mais antiga estável até suas versões superiores
				\begin{itemize}
					\item Google Chrome;
					\item Mozilla Firefox;
					\item Opera;
					\item Microsoft Edge.
				\end{itemize}
		\end{itemize}
	\item Deverá executar em uma máquina contendo, no mínimo, 1GB de memória RAM;
	\item O sistema deverá executar, de maneira persistente , no período de 8h às 18h;
	\item Um programador de manutenção com pelo menos 6 meses de
	experiência deverá ser capaz de dar o suporte necessário para o site;
	\item Ao efetuar o pedido, ele tem que ser confirmado e assim gerando
	o código do mesmo em no máximo 2 segundos; * Existe
	\item Ao efetuar o pedido, os pontos que poderão ser convertidos em cupons denominados
	“Descontos Especiais” na loja de pontos localizada na área “Pontos” deverão chegar em no máximo 24 hora; * Existe
	\item Não mais que 3 em cada 1000 dados de clientes podem ser perdidos
	devido a falhas de software. * Existe
\end{itemize}

\section{Objetivos}
A automatização e modernização das atividades e serviços para a cafeteria Golden Express Coffee Company, como: a inserção de uma loja virtual, sistema de delivery através do site da companhia, implementar um sistema de desconto, inserir uma aba para o cadastro de clientes.

\section{Minimundo}
A “Golden Express Coffee Company”, fundada no dia 18/03/2022, na Rua Mariz e Barros n° 13 é uma cafeteria que tem como objetivo a automatização e modernização de suas atividades e serviços, como: A inserção de uma loja virtual, sistema de delivery através do site da companhia, implementar um sistema de desconto, inserir uma aba para o cadastro de clientes.

A companhia foi inspirada em um clube de amigos que era perito quando o assunto se tratava de café. Após muito projeção, trabalho e especulações, a Golden Express Coffee Company foi fundada com o objetivo de representar o espírito dos apaixonados por café.
O cadastro de clientes é destinado para aqueles consumidores que desejam ser recompensados pela preferência dos nossos produtos, ocorrerá no site da companhia contendo as seguintes informações: Nome, endereço, telefone para contato, e-mail. Após o cadastro, o usuário poderá acessar a página do login para efetuar o mesmo, sendo assim, tendo acesso aos descontos especiais e ao sistema de delivery. Após a realização do cadastro, o usuário terá acesso a uma área denominada “Opções”, lá será possível a alteração dos dados cadastros.

Sistema de pontos funcionará da seguinte maneira, o cliente que possui cadastro no site receberá pontos cada vez que ele realizar uma compra na cafeteria. Essa pontuação poderá ser convertida em cupons denominados “Descontos Especiais” na loja de pontos localizada na área “Pontos”. O usuário logado também terá o acesso ao histórico de compras.

Os "Descontos Especiais" estarão disponíveis somente para os clientes que possuem cadastro em nosso site, são basicamente cupons com a finalidade de diminuir o preço do produto comprado, podendo ser de R\$  5.00, R\$ 10.00, R\$ 20.00 ou até de R\$ 40.00 OFF. Para o uso do benefício, o comprador deverá mostrar o cupom no momento do pagamento para o atendente ou, no caso de um delivery, o desconto especial deverá ser informado antes do pagamento na área “Insira seu Cupom”.

As vendas podem ser realizadas na loja física por um funcionário, ou feitas por via do site (Delivery). A transação ocorre por via de cartões, sendo ele de crédito ou débito, pix ou dinheiro em espécie. Os produtos vendidos serão: Salgados, doces, sucos, chás, pães, pó de café, chaveiros e bonés personalizados.

O sistema de Delivery ocorrerá da seguinte maneira, através do site da companhia, o usuário logado terá o acesso ao catálogo de itens, após selecionar os produtos desejados, o cliente será direcionado a página de pagamento, lá deverá ocorrer o pagamento através de uma API. Após preencher todos os campos, na área de descrição do pedido, o sistema confirmará se o endereço cadastrado está correto e perguntará o nome da pessoa que receberá a entrega. Após confirmar o pedido, o usuário receberá um comprovante de compra e o horário previsto para a entrega de seus itens. É importante ressaltar que o entregador deverá apresentar o recibo da compra na loja após o serviço.