%% -*- coding: utf-8; -*-

\documentclass[
  phd,
  %% master
  %% brazilian
  american
]{Iserj}


%%%
%%% Additional Packages
%%%

  %% \usepackage[brazilian]{babel}      %% in ThesisPUC.cls
  %% \usepackage[utf8]{inputenc}        %% .
  %% \usepackage[T1]{fontenc}           %% .
  %% \usepackage{lmodern}               %% .
  %% \usepackage[pdftex]{graphicx}	%% .

  \usepackage{tabularx}
  \usepackage{multirow}
  \usepackage{multicol}
  \usepackage{colortbl}
  \usepackage[%
    dvipsnames,
    svgnames,
    x11names,
    fixpdftex
  ]{xcolor}
  \usepackage{numprint}
  \usepackage{textcomp}
  \usepackage{booktabs}
  \usepackage{amsmath}
  \usepackage{enumitem}
  \usepackage{amssymb}
  \usepackage{textcomp}
% \usepackage{etoolbox}

%% numprint 
\npthousandsep{.}
\npdecimalsign{,}


%% \tablesmode{figtab} %% [nada, fig, tab ou figtab]


%%%
%%% Counters
%%%

%% uncomment and change for other depth values
%% \setcounter{tocdepth}{3}
%% \setcounter{lofdepth}{3}
%% \setcounter{lotdepth}{3}
%% \setcounter{secnumdepth}{3}


%%%
%%% New commands and other global definitions
%%%

\input{defs}


%%%
%%% Misc.
%%%

\usecolour{true}


%%%
%%% Titulos
%%%

\authorA{Daniel Lima de Souza}
\authorB{Guilherme Gomes da Silva}
\authorC{Gustavo de Menezes Antunes}
\authorD{João Emanoel Nunes de Medeiros}
\authorE{João Victor Pedro Padilha}
\authorF{Lucas Lindoso Paz Domingues}
\authorG{Thales Gomes Santana Santos}
\authorH{Ana Isabel de Melo Mesquita}

\authorRA{de Souza, Daniel Lima}
\authorRB{da Silva, Guilherme Gomes}
\authorRC{Antunes, Gustavo de Menezes}
\authorRD{de Medeiros, João Emanoel Nunes}
\authorRE{Padilha, João Victor Pedro}
\authorRF{Paz Domingues, Lucas Lindoso}
\authorRG{Santana Santos, Thales Gomes}
\authorRH{de Melo Mesquita, Ana Isabel}

\advisorPF{Sancrey Rodrigues Alves}
\advisorMD{Marcus Vinicius dos Santos Claro}
\advisorLTP{Antônio Eduardo Brivio}
\advisorWEB{Antônio Eduardo Brivio}




\title{Golden Express Coffee Company}

\titleuk{Development of a digital microscopy system for automatic
  classification of hematite types in iron ore}

%% \subtitulo{Aqui vai o subtitulo caso precise}

\day{17}
\month{Junho}
\year{2022}

\city{Rio de Janeiro}
\CDD{620.11}
\university{Instituto Superior de Educação do Rio de Janeiro}
\uni{ISERJ}


%%%
%%% Jury
%%%

\jury{%
  \jurymember{Sancrey Rodrigues Alves}{Prof.}
    {Projeto Final}{}
  \jurymember{Marcus Vinicius dos Santos Claro}{Prof.}
    {Modelagem de Dados}{}
  \jurymember{Antônio Eduardo Brivio}{Prof.}
    {Linguagem Técnica de Programação}{}
  \jurymember{Antônio Eduardo Brivio}{Prof.}
    {Programação para WEB}{}
  \schoolhead{Arnaldo Pereira de Carvalho Filho}{Prof.}
}


\acknowledgment{%
  Agradece-se aos docentes Sancrey Rodrigues Alves (Professor Orientador), Marcus Vinicius dos Santos Claro (Professor de Modelagem de Dados 3) e Antônio Eduardo Brivio (Professor de Linguagem e Técnica de Programação 3) pelas
orientações e aconselhamentos para a realização deste trabalho de conclusão do Curso Técnico em Informática.
}


\catalogprekeywords{%
  \catalogprekey{Curso Técnico em Informática}%
}


\keywords{%
  \key{Estudo de Caso;}
  \key{Análise de Requisitos;}
  \key{UML;}
  \key{Programação para WEB;}
}

\chavesuk{%
	\chave{Study of Use Case;}%
\chave{Requirement Analysis;}%
\chave{UML;}%
\chave{WEB Programming;}%
}

\abstract{%
	
	Este trabalho de conclusão de curso - TCC - quantitativamente tem por objetivo a
	automatização e modernização das atividades da cafeteria Golden Express Coffee Comany.
	
	Este documento é formado por todas as fases precisas para a feitura deste sistema, além do estudo do caso de uso por meio de seu diagrama, a síntese da análise de requisitos separadas com o propósito do
	sistema, o minimundo e a lista de requesitos. Também estão incluindos a parte de UML e todo o
	operacional da programação para Java e Web, sendo os seus tópicos descritos separadamente em
	capítulos.
}

\abstractuk{%
	
	This course conclusion work - TCC - aims to automate and modernize the activities of the Golden Express Coffee Comany coffee shop.
	
	This document is formed by all the phases as necessary for the making of the system, in addition to the study of the use case through its diagram, a synthesis of the analysis of separate requirements with the purpose of the system, the miniworld and the list of requirements. They are also including the UML part and the entire operational programming for the Java Web, being presented and explored in chapters.
}


%%%
%%% Dedication
%%%

%\dedication{%
%  Dedicatória do\\ grupo.
%}

%%%
%%% Epigraph
%%%

%\epigraph{%
%  My beautifull epigraph
%}
%\epigraphauthorA{Wassily Kandinsky}
%\epigraphbook{Regards sur le passé}


%%%
%%% 
%%%

\begin{document}

%  \input{abrevs}
  % -*- coding: utf-8; -*-

\chapter{Análise de Requisitos}

\section{Requisitos Funcionais}
\begin{itemize}
	\item Efetuar login;
	\item Realizar cadastro de controle de estoque;
	\item Minimizar ao máximo o número de entregas atrasadas; * Perguntar Vínicus
	\item Inserir dados cadastrais do cliente em formulário próprio;
	\item Permitir a busca itens como cafés, doces e salgados específicos contidos no cardápio da loja;
	\item Adicionar itens como cafés, doces e salgados específicos contidos no cardápio da loja;
	\item Receber pontos por compras realizadas;
	\item Trocar pontos por cupons na sessão "Descontos Especiais" do site;
	\item Permitir a utilização de cupons antes de efetuar os pedidos;
	\item Realizar pedidos, escolhendo a forma de pagamento desejada;
	\item Comunicar-se com a/o atendente da loja.
\end{itemize}

\section{Requisitos Não Funcionais}
\begin{itemize}
	\item O sistema deverá executar
		\begin{itemize}
			\item nos Sistemas Operacionais de 64 bits da sua versão estável mais antiga até sua versão superior
				\begin{itemize}
					\item Windows XP e mais atuais;
					\item Linux nas suas Distribuições Ubuntu e Debian;
					\item MacOS.
				\end{itemize}
			\item nos navegadores da versão mais antiga estável até suas versões superiores
				\begin{itemize}
					\item Google Chrome;
					\item Mozilla Firefox;
					\item Opera;
					\item Microsoft Edge.
				\end{itemize}
		\end{itemize}
	\item Deverá executar em uma máquina contendo, no mínimo, 1GB de memória RAM;
	\item O sistema deverá executar, de maneira persistente , no período de 8h às 18h;
	\item Um programador de manutenção com pelo menos 6 meses de
	experiência deverá ser capaz de dar o suporte necessário para o site;
	\item Ao efetuar o pedido, ele tem que ser confirmado e assim gerando
	o código do mesmo em no máximo 2 segundos; * Existe
	\item Ao efetuar o pedido, os pontos que poderão ser convertidos em cupons denominados
	“Descontos Especiais” na loja de pontos localizada na área “Pontos” deverão chegar em no máximo 24 hora; * Existe
	\item Não mais que 3 em cada 1000 dados de clientes podem ser perdidos
	devido a falhas de software. * Existe
\end{itemize}

\section{Objetivos}
A automatização e modernização das atividades e serviços para a cafeteria Golden Express Coffee Company, como: a inserção de uma loja virtual, sistema de delivery através do site da companhia, implementar um sistema de desconto, inserir uma aba para o cadastro de clientes.

\section{Minimundo}
A “Golden Express Coffee Company”, fundada no dia 18/03/2022, na Rua Mariz e Barros n° 13 é uma cafeteria que tem como objetivo a automatização e modernização de suas atividades e serviços, como: A inserção de uma loja virtual, sistema de delivery através do site da companhia, implementar um sistema de desconto, inserir uma aba para o cadastro de clientes.

A companhia foi inspirada em um clube de amigos que era perito quando o assunto se tratava de café. Após muito projeção, trabalho e especulações, a Golden Express Coffee Company foi fundada com o objetivo de representar o espírito dos apaixonados por café.
O cadastro de clientes é destinado para aqueles consumidores que desejam ser recompensados pela preferência dos nossos produtos, ocorrerá no site da companhia contendo as seguintes informações: Nome, endereço, telefone para contato, e-mail. Após o cadastro, o usuário poderá acessar a página do login para efetuar o mesmo, sendo assim, tendo acesso aos descontos especiais e ao sistema de delivery. Após a realização do cadastro, o usuário terá acesso a uma área denominada “Opções”, lá será possível a alteração dos dados cadastros.

Sistema de pontos funcionará da seguinte maneira, o cliente que possui cadastro no site receberá pontos cada vez que ele realizar uma compra na cafeteria. Essa pontuação poderá ser convertida em cupons denominados “Descontos Especiais” na loja de pontos localizada na área “Pontos”. O usuário logado também terá o acesso ao histórico de compras.

Os "Descontos Especiais" estarão disponíveis somente para os clientes que possuem cadastro em nosso site, são basicamente cupons com a finalidade de diminuir o preço do produto comprado, podendo ser de R\$  5.00, R\$ 10.00, R\$ 20.00 ou até de R\$ 40.00 OFF. Para o uso do benefício, o comprador deverá mostrar o cupom no momento do pagamento para o atendente ou, no caso de um delivery, o desconto especial deverá ser informado antes do pagamento na área “Insira seu Cupom”.

As vendas podem ser realizadas na loja física por um funcionário, ou feitas por via do site (Delivery). A transação ocorre por via de cartões, sendo ele de crédito ou débito, pix ou dinheiro em espécie. Os produtos vendidos serão: Salgados, doces, sucos, chás, pães, pó de café, chaveiros e bonés personalizados.

O sistema de Delivery ocorrerá da seguinte maneira, através do site da companhia, o usuário logado terá o acesso ao catálogo de itens, após selecionar os produtos desejados, o cliente será direcionado a página de pagamento, lá deverá ocorrer o pagamento através de uma API. Após preencher todos os campos, na área de descrição do pedido, o sistema confirmará se o endereço cadastrado está correto e perguntará o nome da pessoa que receberá a entrega. Após confirmar o pedido, o usuário receberá um comprovante de compra e o horário previsto para a entrega de seus itens. É importante ressaltar que o entregador deverá apresentar o recibo da compra na loja após o serviço.
  % -*- coding: utf-8; -*-

\chapter{UML}
\section{Diagramas}

\begin{figure}[!h]
	\centering
	\includegraphics[width=15cm]{DER}
	\caption{Diagrama de Entidade e
		Relacionamento da Golden Express Coffee Company}
\end{figure}

\begin{figure}[!h]
	\centering
	\includegraphics[width=15cm]{DCU}
	\caption{Diagrama de Caso de Uso da Golden Express Coffee Company}
\end{figure}

\begin{figure}[!h]
	\centering
	\includegraphics[width=15cm]{DC}
	\caption{Diagrama de Classes da Golden Express Coffee Company}
\end{figure}

  % -*- coding: utf-8; -*-

\chapter{Descrição do Sistema}
\section{Tela Inicial do Sistema}
\begin{figure}[!h]
	\centering
	\includegraphics[width=15cm]{Cabeçalho}
	\caption{Cabeçalho}
\end{figure}

\begin{figure}[!h]
	\centering
	\includegraphics[width=15cm]{Sessão Início}
	\caption{Sessão Início}
\end{figure}

\begin{figure}[!h]
	\centering
	\includegraphics[width=15cm]{Sessão Sobre}
	\caption{Sessão Sobre}
\end{figure}

\begin{figure}[!h]
	\centering
	\includegraphics[width=15cm]{Sessão Cardápio}
	\caption{Sessão Cardápio}
\end{figure}

\begin{figure}[!h]
	\centering
	\includegraphics[width=15cm]{Sessão Contato}
	\caption{Sessão Contato}
\end{figure}

\begin{figure}[!h]
	\centering
	\includegraphics[width=15cm]{Rodapé}
	\caption{Rodapé}
\end{figure}

\section{Funcionalidade de Login}
  % -*- coding: utf-8; -*-

\chapter{Código fonte do sistema}
\section{Tela Inicial do Sistema}
\begin{figure}[!h]
	\centering
	\includegraphics[width=15cm]{Contato HTML 1}
	\caption{Contato HTML 1}
\end{figure}

\begin{figure}[!h]
	\centering
	\includegraphics[width=15cm]{Contato HTML 2}
	\caption{Contato HTML 2}
\end{figure}

\begin{figure}[!h]
	\centering
	\includegraphics[width=15cm]{Contato CSS 1}
	\caption{Contato CSS 1}
\end{figure}

\begin{figure}[!h]
	\centering
	\includegraphics[width=15cm]{Contato CSS 2}
	\caption{Contato CSS 2}
\end{figure}

\begin{figure}[!h]
	\centering
	\includegraphics[width=15cm]{Contato CSS 3}
	\caption{Contato CSS 3}
\end{figure}

\begin{figure}[!h]
	\centering
	\includegraphics[width=15cm]{Contato CSS 4}
	\caption{Contato CSS 4}
\end{figure}

\begin{figure}[!h]
	\centering
	\includegraphics[width=15cm]{Contato CSS 5}
	\caption{Contato CSS 5}
\end{figure}

\begin{figure}[!h]
	\centering
	\includegraphics[width=15cm]{Contato CSS 6}
	\caption{Contato CSS 6}
\end{figure}

\begin{figure}[!h]
	\centering
	\includegraphics[width=15cm]{Contato CSS 7}
	\caption{Contato CSS 7}
\end{figure}

\begin{figure}[!h]
	\centering
	\includegraphics[width=15cm]{Contato CSS 8}
	\caption{Contato CSS 8}
\end{figure}

\begin{figure}[!h]
	\centering
	\includegraphics[width=15cm]{Contato CSS 9}
	\caption{Contato CSS 9}
\end{figure}

\begin{figure}[!h]
	\centering
	\includegraphics[width=15cm]{Contato PHP 1}
	\caption{Contato PHP 1}
\end{figure}

\begin{figure}[!h]
	\centering
	\includegraphics[width=15cm]{Contato PHP 2}
	\caption{Contato PHP 2}
\end{figure}

\section{Funcionalidade de Login}
  % -*- coding: utf-8; -*-

\chapter{Conclusions}

Aqui entram as conclusões...

  %% ...
  \arial
  \bibliography{tiny}
  \normalfont
  \input{appendix}

\end{document}
