% -*- coding: utf-8; -*-

\chapter{Introdução}

O ferro é o metal mais usado pela sociedade devido à alta
disponibilidade, pelas propriedades físicas (dutilidade,
maleabilidade, resistência mecânica, etc.), por sua importância na
produção de aço e ferro fundido, assim como pelas suas muitas
aplicações.

O uso do ferro vem desde a antiguidade. Provavelmente a primeira vez
que o homem fez contato com o ferro metálico foi sob a forma de
meteoritos, daí a etimologia da palavra siderurgia, cujo radical
latino \textit{sider} significa estrela ou astro. No antigo Egito
foram descobertos ornamentos de ferro datados de cerca de 4000 A.C.,
também na pirâmide de Gizé foram achadas peças datadas de 2900 A.C. A
primeira indústria do ferro apareceu ao sul do Cáucaso, 1700 A.C.,
entre os Hititas. Também, na Assíria, foram encontradas ferramentas de
aço que datavam de 700 anos A.C. \cite{1}

Na atualidade, o uso crescente do aço e do ferro fundido na fabricação
de produtos de consumo evidencia a importância da indústria
metalúrgica para a economia nacional e global. Em contrapartida, a
qualidade do minério de ferro disponível vem diminuindo ao longo dos
anos. Portanto, as empresas de mineração tem se esforçado para
aumentar a produção e melhorar seus produtos, a fim de manter e
aperfeiçoar o seu desempenho no mercado.

O minério de ferro é um material policristalino que passou por vários
processos naturais complexos. Estes processos ocorreram durante tempos
geológicos, devido aos efeitos da pressão, às mudanças de temperatura,
à recristalização e à erosão, dando origem a diversas características
intrínsecas, e consequentemente, a um comportamento industrial
variável.\cite{2}

Assim, o minério de ferro é normalmente utilizado de duas formas:
minérios granulados e minérios aglomerados. Os granulados (entre $25$
mm e $6$ mm) são adicionados diretamente nos fornos de redução,
enquanto os aglomerados são os minérios finos que devido à sua
granulometria necessitam de aglomeração. Daí surgem as denominações
\textit{sinter-feed} (entre $6,35$ mm e $0,15$ mm) e
\textit{pellet-feed} (menos de $0,15$ mm) que identificam as frações
usadas nos processos de sinterização e pelotização,
respectivamente.\cite{3}

Devido ao grande interesse econômico, bem como o seu desempenho
durante o processo, a caracterização de cada uma destas frações
adquire grande importância. No entanto, não existe um método universal
para isto. De fato, este é um problema complexo, uma vez que são
diversos os atributos que caracterizam forma, textura, trama ou
porosidade, assim como as maneiras como são combinados em cada
caso.\cite{4}

Os minerais carreadores de ferro mais comuns (hematita, magnetita e
goethita) podem ser identificados visualmente no Microscópio de Luz
Refletida (MLR) através de suas refletâncias distintas.\cite{5} Assim,
a caracterização qualitativa de minérios de ferro é geralmente
realizada através de avaliação visual no MLR.

Na indústria mineral, a caracterização microestrutural (mineralógica e
textural) do minério de ferro e seus aglomerados é tradicionalmente
realizada por operadores humanos, pela observação de amostras ao MLR,
para identificar as fases presentes e estimar suas frações. Esse é um
procedimento rotineiro, realizado algumas vezes por dia e
consequentemente suscetível a falhas decorrentes da fadiga humana,
além de erros aleatórios diversos. Deste modo, tem havido um interesse
crescente no desenvolvimento de sistemas automáticos de análise
quantitativa que possam conferir maior reprodutibilidade,
confiabilidade e velocidade.

Por outro lado, sistemas automáticos de análise digital de imagens são
capazes de identificar hematita, magnetita e goethita pelas suas
tonalidades em imagens obtidas pelo MLR. Estes sistemas têm a vantagem
de serem mais velozes, práticos e reprodutíveis do que um operador
humano. Nos últimos anos, algumas metodologias foram desenvolvidos
para realizar a caracterização mineralógica de minérios de ferro
através de sistemas de análise de imagens.\cite{6,7,8,9}

A grande maioria dos minérios de ferro brasileiros é essencialmente
hematítica, geralmente envolvendo outros minerais como magnetita,
goethita e minerais de ganga, principalmente quartzo. No entanto,
estes minérios apresentam grande diversidade de microestruturas. A
hematita, por exemplo, pode ser classificada como lobular, lamelar,
granular, microcristalina ou martita.

O tamanho, a forma e a distribuição dos cristais de hematita podem
influenciar na redutibilidade e resistência mecânica dos
aglomerados. Por exemplo, hematitas granular e lamelar aumentam a
resistência mecânica dos aglomerados, mas reduzem sua porosidade e sua
redutibilidade. Já a hematita martítica age no sentido oposto,
aumentando a porosidade e redutibilidade dos aglomerados, mas
reduzindo sua resistência mecânica.\cite{2,10,11} Assim, a
determinação das características texturais da hematita certamente
contribui para um melhor conhecimento dos minérios de ferro, abrindo
novas possibilidades a fim de aprimorar seu processamento.\cite{12}

A hematita é um mineral fortemente anisotrópico. Ela apresenta
1pleocroísmo de reflexão (birrefletância), ou seja, sua refletância e,
consequentemente, o seu brilho na imagem mudam com diferentes
orientações dos cristais em relação ao plano de incidência da
luz.\cite{5} Essa variação de brilho é sutil, mas é perceptível ao
olho humano treinado no MLR.

Por sua vez, o uso combinado de um polarizador e um analisador no MLR
gera variações de brilho e cores devido à anisotropia.\cite{13} Esta
abordagem pode ser usada para obter imagens que apresentam um
contraste suficiente para diferençar os cristais.

Como já constatado, os minérios de ferro podem ter uma estrutura muito
complexa, com a associação de diferentes minerais e texturas. De tal
modo, não é muito difícil imaginar que criar um algoritmo de análise
de imagens capaz de identificar e caracterizar todas as formas de
hematita é um grande desafio. Sendo assim, este trabalho tem como
objetivo desenvolver uma metodologia de aquisição, processamento e
análise de imagens no MLR para:

\begin{enumerate}[label=(\roman{*})]
  \item Segmentar cristais de hematita compacta (granular, lamelar e
    lobular) no minério de ferro em amostras ricas neste mineral;
  \item Medir tamanho de cristais de hematita compacta e;  
  \item Medir os cristais de hematita compacta visando a sua
    classificação segundo sua morfologia.
\end{enumerate}

O grupo de pesquisa em Microscopia Digital (MD) do DEMa/PUC-Rio está
tentando desenvolver uma metodologia de classificação automática de
tipos de hematita através de duas abordagens diferentes, analítica e
sintética. A primeira é objeto de estudo deste trabalho, já a segunda
será desenvolvida por outro integrante do grupo como parte da sua
dissertação de mestrado.

O método sintético consiste em empregar parâmetros de textura com o
objetivo de identificar tipos texturais de hematita não compacta
(martita e microcristalina). Com este fim, as imagens são analisadas
em \textit{textels} (elementos de textura), dos quais são extraídos os
parâmetros de textura. Estes parâmetros de textura são empregados como
atributos no sistema de classificação.

Por sua vez, o método analítico combina diversas imagens de um mesmo
campo, obtidas com e sem polarização. A imagem sem polarização permite
separar a hematita das demais fases, a partir de seu brilho. As
imagens com polarização permitem encontrar as fronteiras entre os
cristais de hematita. Uma vez separados os cristais de hematita, estes
são medidos e classificados segundo sua morfologia.\cite{2,13}

O método analítico tem como característica fundamental o
reconhecimento individual de cada cristal de hematita, porém o método
é ineficiente na identificação dos cristais das fases não compactas. É
assim que o método sintético visa complementar esta deficiência,
criando, no conjunto, uma metodologia de classificação automática dos
cinco tipos de hematita.

A presente tese está organizada em cinco capítulos. O primeiro
capítulo consiste desta introdução.

O segundo capítulo (``Revisão Bibliográfica") traz uma retrospectiva
do minério de ferro, sua importância econômica, sua composição e sua
microestrutura. Ao mesmo tempo, expõe o conteúdo teórico das diversas
técnicas experimentais envolvidas no trabalho e descreve algumas
técnicas que vêm sendo estudadas.

O terceiro capítulo (``Materiais e Métodos") descreve o mecanismo de
preparação das amostras de minério de ferro. Por sua vez, apresenta as
etapas experimentais, assim como os equipamentos e técnicas usadas na
análise destas amostras. Neste capítulo, são também descritas as
técnicas de identificação, medição e classificação dos cristais de
hematita.

O quarto capítulo (``Resultados e Discussões") apresenta e discute os
resultados. Neste capítulo, são expostas as vantagens e desvantagens
de cada técnica experimental, assim como sua eficiência na
identificação de cristais de hematita.

Finalmente, o quinto capítulo (``Conclusões") apresenta as conclusões
e propostas para trabalhos futuros.

\textit{Cumpre comentar, nesta introdução, sobre uma característica um
  tanto peculiar da estrutura do presente trabalho. Para atingir os
  objetivos listados acima foram desenvolvidos diferentes métodos de
  microscopia digital, envolvendo a criação de rotinas sofisticadas de
  processamento e análise de imagens. Assim, parte dos métodos
  "utilizados" foram, na verdade, "desenvolvidos" no decorrer do
  trabalho. Por esta razão, os conteúdos dos capítulos de Materiais e
  Métodos e de Resultados muitas vezes se misturam e retroalimentam,
  gerando uma estrutura um tanto fora do padrão.}
